\documentclass[11pt, a4paper]{report}
\usepackage{graphicx}
\graphicspath{ {/home/xrand/Documents/LaTex/} }
\usepackage{amsmath}
\usepackage{enumerate}
\usepackage{gensymb}

\title{ {Estimation of Site-Specific Biomass and Leaf Area Index in a Young Scots pine Stand in Southern Sweden}\\ {Swedish University of Agricultural Sciences}\\ }
\author{Olamide Michael Adu}
% add image to this part
\date{2 June 2023}

\setcounter{tocdepth}{4}
\begin{document}
\maketitle
\begin{abstract}

Accurate biomass and leaf area index (LAI) estimation are crucial for understanding forest productivity and carbon dynamics. Models are crucial to estimating forest biomass and LAI. This study aimed to estimate site-specific aboveground biomass and LAI in a young Scots pine stand in southern Sweden. Destructive sampling was carried out to obtain data for biomass and LAI. Site-specific models were developed, including an individual tree LA model based on diameter at breast height (DBH), LAI models based on stand density (LAImodel1) and basal area (LAImodel2), stand-level biomass models developed based on stand density and basal area as independent variables, and tree-level biomass models developed based on DBH. Directly measured LAI values were used to validate a generalized LICOR model for indirect LAI estimation, which was compared to the site-specific LAI model developed in this study. Evaluation metrics such as mean absolute error (MAE), mean squared error (MSE), and root mean squared error (RMSE) were used to compare the site-specific and generalized model. Results showed that site-specific LAI models of this study gave reasonable estimates (LAImodel1 have MAE = 0.166, MSE = 0.039, RMSE = 0.198, and LAImodel2 – MAE = 0.011, MSE = 0.000, RMSE = 0.015) compared to the generalized model (MAE = 0.66, MSE = 0.435, RMSE = 0.66). Directly measured LAI and indirectly measured LAI using the generalized LICOR model had a weak correlation (R2 = 0.1339). For stand-level biomass models, stand density showed moderate correlation (R2 = 0.49) to the total stand biomass, and basal area was strong correlation (R2 = 0.75) with the total stand biomass. These findings demonstrate the importance of using site-specific models to reduce prediction errors when estimating biomass and LAI in forest stands.

\textit{Keywords}: Scots pine, Biomass estimation, Leaf area index, Model Validation, Site-specific models
\end{abstract}
\tableofcontents

\chapter{Introduction}
\section{The Concept of Forest Biomass}
Biomass is a term that has been used across different fields, including agricultural sciences, renewable energy, biological sciences, environmental sciences, and forestry, to
mean almost the same thing with little alterations. The difference in the meaning across
these fields may be regarding their origin, end-use, or composition of the organic matter.
Biomass, in simple terms, refers to the total mass of organic matter produced by living
organisms (West 2014; Kumar et al. 2021; Page-Dumroese et al. 2022). This organic matter
can come from different sources, including plants, animals, and microorganisms
(Roberts et al. 2015; Wasmi \& Salih 2021). In forestry, the biomass of trees is the focus, and it includes all parts of a tree, like the branches, stem, leaves, needles, stumps, roots, and fine roots (Page-Dumroese et al. 2022).

Forest biomass is the most abundant biomass found on land, having about 70 - 90\% of the terrestrial biomass (Houghton 2008). This biomass is essential to the survival of humans and animals alike. For example, in Sweden, forest biomass for timber, pulp, and paper production accounts for 73.5\% of the biomass used within the country (Kumar et al. 2021). The importance of forest biomass is not limited to this. It is used as part of nature-based solutions to achieve the Sustainable Development Goals (SDGs), especially goal 13 – climate  action, which is gaining much attention (Millennium Ecosystem 2005; IPCC 2022). Using the forest as anSDG tool is closely associated with the range of ecosystem services that it  provides to both the people and the planet (Millennium Ecosystem 2005). For example, forest biomass is a source of foreign exchange for countries like Canada, Finland, Brazil, Sweden, The Russian Federation, the United States, and China, to mention a few (Oishimaya 2017). In Sweden, the forest industry exports more than 85\% of its products, making it the fifth largest exporter of pulp, paper, and sawn timber worldwide. They also contribute to Sweden’s  national GDP and provide more than 100,000 jobs (Hallsten \& Heinsoo 2016).

\subsection{Factos Affecting Forest Biomass Production}
The different services the forest provides make it necessary to maximize the forest and increase its productivity as much as possible. An increased understanding of the requirements and characteristics of tree biology by researchers and forest managers is essential to optimize land use and increase forest productivity. To do so, foresters/forest managers need to understand the requirements and characteristics of tree biology. 

Some conditions are necessary for tree growth. These requirements are optimum sunlight, carbon dioxide (CO2) from the atmosphere, well-aerated soil with optimum temperature, water, and nutrients from the ground (West 2014). The combination of these factors enables the process of photosynthesis possible. Simply put, photosynthesis is the process by which plants  produce food. Its food increases its biomass, i.e., growth (West 2014). Silvicultural practices are another factor that can impact the production of forest biomass. While not applied in the wilderness or primary forest, they are used in close-to-nature or nature-based forest types and are common in forest plantations (Larsen 2012). These practices are in three stages: pre-planting, planting, and post-planting operations. Examples of such silvicultural practices are thinning, precommercial thinning, breeding and genetics, and soil preparation to mention a few. The practices influence the survival, growth, development, quantity, and quality of tree biomass found in any forest plantation for a given period (Larsen 2012).

\subsection{Biomass Estimation}
Scots pine (\textit{Pinus sylvestris L.}) is an important tree species in Sweden, and its economic importance is well established (Lula et al. 2021). Numerous studies have been conducted to estimate and develop biomass functions for Scots pine in Sweden. Biomass functions should meet some requirements before being used. They should give reliable estimates of the area of concern. They should be based on variables that are easy to estimate and can be reliably collected from inventory data (Repola 2009; Repola \& Ahnlund Ulvcrona 2014). These functions are derived through destructive, non-destructive sampling, or both. 

Biomass estimation studies using destructive sampling often have smaller sample sizes and simpler models (Weiskittel et al. 2011). However, larger sample sizes have been used in some studies, such as those conducted by Repola (2009), Repola and Ahnlund Ulvcrona (2014), and Marklund (1988), with the latter being used mainly in Sweden. Marklund (1988)’s biomass  equations for pine, birch, and spruce used 1286 tree samples spread around 131 sites in Sweden. The primary parameter in biomass estimation at the individual tree level is the diameter (DBH), although height can also be used in addition to DBH. However, using DBH alone assumes that the relationship between DBH and height is fixed, which is not always the case, as species, site conditions, and management history can also affect tree height (Weiskittel et al. 2011). Therefore, using DBH alone should not be considered when estimating regional or national biomass, as it neglects species-specific variability and site effects (Zianis \& Mencuccini 2004).

\section{Leaf Area Index (LAI) and Its Significance in Forest Productivity Measurement}
The leaf area index (LAI) is a crucial parameter for measuring the productivity of
forest ecosystems (Binkley \textit{et al.} 2004; Stape \textit{et al.} 2008). It reflects the photosynthetic capacity of a forest canopy by measuring the total leaf surface area per unit of ground surface area in a forest stand (West 2015). The growth rate of forests is proportional to the amount of sunlight their leaves intercept (Stape \textit{et al.} 2008; Binkley \textit{et al.} 2010). The larger the LAI of a stand, the more sunlight it can intercept and utilize for photosynthesis (Almeida \textit{et al.} 2007; Guiterman \textit{et al.} 2012). Although leaves are the chief photosynthetic organ in trees, other parts of
trees, such as the petioles, green flowers, cones, and stem tissue, are capable of
photosynthesizing (Pfanz \textit{et al.} 2002). However, the photosynthetic rate by non-
leaf parts is limited by tree age, especially the stem of trees. This type of
photosynthesis involving recycling CO2 by recapturing respiratory CO2 before it
diffuses out of the stem is called corticular photosynthesis. Reabsorption of CO2 by
tree parts can compensate for 60 – 90\% of potential carbon loss due to respiration
(Pfanz \textit{et al.} 2002).

\section{Role of LAI in Understanding Ecosystem Processes and Adaptations in Plantations}
The forest canopy is a critical driver in ecosystem processes and biomass production (Selin 2019). LAI is an essential parameter for understanding the ecophysiological processes of stand canopy (West 2014; West 2015). Stress in trees can be determined by using LAI. During periods of limited water availability and drought, plantations may reduce their water loss by decreasing their LAI as an adaptation strategy. Trees do this by shedding their leaves. Some studies have shown that LAI varies annually and seasonally (Guiterman \textit{et al.} 2012). Thinning can also affect LAI, which leads to an immediate reduction in LAI at the stand level. While trees left in the stand may adjust their photosynthetic capacity over time, the stand may not recover to its original LAI (Guiterman \textit{et al.} 2012). Pruning, on the other hand, has been found to have little effect on LAI since the leaves in the canopy shade themselves. Even with reduced leaf area, plantations can still intercept about 80\% of their sunlight (Alcorn \textit{et al.} 2008; West 2014). Initially, leaves do not function at their maximum capacity, and pruning allows the remaining leaves to increase their photosynthetic capacity when reduced (Quentin \textit{et al.} 2011).
\section{Forest Resource Management Models in Sweden}
Models are essential tools for managing forest resources; they are abstractions of reality attempting to conceive some relationships of a system (Weiskittel \textit{et al.} 2011). In Sweden, models have a long history. They are developed to understand and simulate forest development, support decision-making, and evaluate management strategies. There are three types of models, viz, empirical, process-based and hybrid models. Empirical models are based on statistical correlations and are one of Sweden's most used models. The process-based model utilizes ecophysiological processes that influence growth rather than models based on statistical correlations. The hybrid models combine some ecophysiological process data and statistical correlations to predict stand development (Weiskittel \textit{et al.} 2011; Appiah Mensah \textit{et al.} 2020). Empirical models uses data that are easy to obtain compared to Process-based or Hybrid model types. They are easy to apply in practice having a wide range of applications with good predictive capabilities (Weiskittel \textit{et al.} 2011).

Models, data, and analytical methods are integrated together to make software tools or applications that assist forestry professionals in making informed decisions regarding forest management. These software tools or applications are called Decision Support Systems (DSS) (Söderberg \& Lundström 1996). Hugin’s system is an example of a DSS that uses many empirical-based models (Söderberg \& Lundström 1996; Mats 2015). Developed in the 70s, it incorporates numerous models to predict timber production and quality, biomass, costs, and revenue under different scenarios of forest management strategies (Söderberg \& Lundström 1996). The Heureka DSS has replaced Hugin’s system and boasts more capabilities than its predecessor (Elfving 2010; Mats 2015). The Heureka DSS covers a broader aspect of forestry beyond timber management. Its robust infrastructure allows forest analysis, planning, and management from a stand level to a regional level (Wikström \textit{et al.} 2011). Heureka contains different packages used to analyze, plan, visualize, and generate numerous scenarios based on the set rules of a management system. Models in Heureka are, in one way or another, related to tree development and treatments (silvicultural practice) employed. Also, it allows new models to be added, making the system continually evolving (Wikström \textit{et al.} 2011; Mats 2015). Therefore, empirical studies like this one are important for improving such systems.
\section{Research Objectives and Justification}
LAI and biomass models have been created using destructive or/and non-destructive sampling methods (Marklund 1988; Goude \textit{et al.} 2019; Appiah Mensah \textit{et al.} 2020; Cohrs \textit{et al.} 2020; Pertille \textit{et al.} 2020; Vafaei \textit{et al.} 2021). These models are usually generalized and are based on broad assumptions resulting in inaccurate predictions for some sites. These inaccurate predictions might be due to the models not considering site-specific characteristics (Weiskittel \textit{et al.} 2011). Site-specific empirical models could help with accurate predictions, especially when there is limited data or when the focus is on a specific aspect of a system or process. The creation of these models is critical for estimating carbon stocks for carbon accounting purposes and assessing forest productivity with minimal error (Weiskittel \textit{et al.} 2011). 

Therefore, this study is relevant given Sweden’s limited availability of site- specific LAI and biomass models. This research will also compare the newly developed LAI models with an existing generalized model by Goude \textit{et al.} (2019), allowing for an evaluation of the accuracy and applicability of site-specific models. Goude \textit{et al.} (2019) developed multiple models for Scots pine and Norway spruce. Goude \textit{et al.} (2019)’s study compared directly and indirectly measured leaf area measurements of both species. The model of Goude \textit{et al.} (2019) uses a sample size that cut across various sites in Sweden from the north region to the south. Models developed by Goude \textit{et al.} (2019) related to this study are indirectly measured LAI using LAI-2200 C, and directly measured LA and LAI using DBH and Basal area as the explanatory variables respectively. While caution should be applied when applying site-specific models to other forests, their creation is essential for more accurate predictions and sustainable forest management. Thus, the objectives of this study are: 
\begin{enumerate}[i.]
\item To create site-specific LA and LAI functions and compare the function with Goude \textit{et al.} (2019). 
\item To develop a site-specific empirical-based biomass function.
\end{enumerate}

\chapter{Materials and Methods}
\section{Study Area}
The thesis study was conducted in Hallarp, Ljunby Kommun, Kronoberg County, Sweden. The site is an experimental site of Scots pine managed by the Tönnersjöheden research station. It is part of a series of thinning experiments (trial 2020) established in Tönnersjöheden and Siljansfors. The trials are divided into blocks with different future thinning treatments, which include control, thinning from below, heavy thinning from below, and thinning from above (Silvaboreal - Försök, n.d.). The destructive biomass sampling performed for this study was done during winter 2023, prior to thinning treatments establishment.
\section{Tree Selection}
Trees used for this study were chosen before felling, covering a diameter range of 4 cm to 22 cm with a 2 cm interval for each diameter class (e.g., 4 cm – 6 cm, 6 cm – 8 cm, up to 20 cm – 22 cm). Trees with physical deformities or defects, such as large holes or injuries in their stem, were not selected. An on-site visit was conducted to verify the trees’ physical condition and location. If trees had defects, they were replaced with trees in the same diameter class without deformity. Sixteen trees were selected, with two chosen trees per diameter class except for the 4 cm – 6 cm and 20 cm – 22 cm diameter classes, where one tree was selected. This approach was employed to ensure that a representative sample of trees of various sizes was obtained, avoiding overestimation that could occur from selecting too many large trees.
\section{Tree Processing}
\subsection{Field Processing}
Selected trees in the site were marked at 1.3m with a ribbon and cross calipered to get their DBH. The trees were felled afterwards using a chainsaw. A measuring tape was passed straightly along the stem to measure the total height. The total height was measured by placing the 1.3m mark of the measuring tape on the ribbon and passed along the tree length. Additionally, the height at which the living crown started and the length of the living crown were recorded. Afterwards, the living crown was divided into three sections to make up three strata. The first stratum is the position where the first living branch was found, and the third is the last section which includes the top/tip of the tree.

Furthermore, four living branches were selected at each stratum; the selection was made across and along the length of each stratum, hereafter called sample branches. Four tree branches were also chosen from the section below stratum 1; this section is referred to as stratum 0. All sample branches were placed in different plastic bags and carefully labelled according to the plot, tree, and stratum numbers, and their fresh weights were measured. The sample branches were later taken to the laboratory for further processing. In total, there were twelve sample branches per tree. After this, the trees were debranched, and their branches were gathered into a plastic bag and labelled according to their stratum number. For strata 1, 2, and 3, the branches collected were separated into living and dead branches, and their fresh weight was measured.

The length of the ten most recent annual shoots was measured using the carpenter ruler. Subsequently, the whole tree length was marked at 2m intervals, giving markings at 1.3m, 2m, 4m, 6m, 8m, etc. The stem was then cross calipered starting from the base of the stem to the 1.3m and 2m markings, and then the subsequent 2m interval markings to get their diameter measurement along the trunk. The bark thickness at each point, starting from the base, was also measured. Sample disks were collected at each marking, placed in a plastic bag, and labelled accordingly. Afterwards, the lengths and fresh weights of each stem part were taken. The stem parts were disposed of in the forest after this. These procedures were repeated for all the selected trees (Fig 1).

The indirect LAI measurements were done with the LICOR-2200 in two (out of four) blocks close to when the biomass sampling was performed. The measuring points were obtained on two diagonals in each plot, similar to Goude et al. (2019)'s methodology. There were about 50 points per plot, and eight plots were measured in total.
% add image here
\subsection{Laboratory Processing}
For the biomass analysis, the sample disks collected from the stems were dried in an oven at 70\degree C for 24 hours. The barks were separated from the disks, and the bark and sample disks were returned to the oven until a constant weight was achieved. After, the weight of the disks and barks was measured on a precision scale to determine the dry weight (Fig. 2). The bark percentage was calculated by subtracting the dry weight of the bark from the dry weight of the entire disk and then dividing the difference by the dry weight of the entire disk. The biomass of each section was calculated by adding up the dry weights of the disks within each section, including the bark.

The sample branches collected from the field were also processed. The samples were first sorted and prepared for further analysis. Afterwards, 20 needles were collected per branch for each stratum. These 20 needles per branch make it 80 needles collected per stratum. These 80 needles per stratum were placed in a plastic bag, labelled, and put into a freezer for LAI measurement later, further referred to as LAI needles. The remaining sample branches were cut up, set in a metallic frame, labelled, and put into the oven for 24 hours. The sample branches were removed after 24 hours, and their needles were separated, placed in a new metallic frame, and labelled. The sample branches without needles and their needles (hereafter referred to as sample needles) were put into the oven for another 24 hours. After 24 hours, the sample needles and branches without needles were taken out, weighed, and recorded (Fig. 3). 

The LAI needles were digitally scanned, and their area was measured using the ImageJ version 1.53t image analysis software. Afterwards, the LAI needles were oven-dried for 24 hours at 70oC, and their dry weight was measured later.
% add another image here
\section{Data Processing and Analysis}
Data entry and pre-processing were conducted using Microsoft Excel version 2302 (build 16130.20218), while the statistical analyses were performed using R version 4.2.3 (Shortstop Beagle) by R-Core-Team (2023). The data was checked for inaccuracies, inconsistencies, and missing values during the pre-processing phase to ensure data quality. Linear models were fitted using the lm package, and the anova package was used to test for differences between strata. 

Normality and homoscedasticity of the measurement data were tested using the Shapiro-Wilk Test (Shapiro \& Wilk, 1965). A p-value greater than 0.05 indicatedthat the  distribution followed normality. The moment package was used to check the skewness of the distribution.

To estimate Leaf Area Index (LAI), a step-by-step procedure is shown in Fig. 4.
%image here
The analyses of the biomass content of trees, Specific Leaf Area (SLA), and Leaf Area (LA) were conducted first on needles and then at the individual tree level. 

The projected area (PA) in $cm^2$ was calculated as the sum of the LAI needles area obtained from ImageJ according to their plot, tree, and strata.

The half total surface area ($HTSA$, $cm^2$) was predicted using the regression equation ($R2 = 0.997$) by Goude et al., 2019 (Equation 1), as the volume of the sample needles was not measured during the laboratory work. 

The specific leaf area (SLA), measured in $cm2g_1$, was calculated for each LAI needle by dividing the HTSA ($cm^2$) by the dry weight ($g$) of the needles. 

The total needle dry weight ($g$) of each stratum was measured by getting the ratio of the needles weighed in the lab, and the needles weighed on the field were multiplied by the SLA ($cm^2g^{-1}$) to calculate the half total leaf area ($cm^2$) of each stratum.

This information was then summarized by the plot and tree number to obtain each tree’s half-total leaf area ($m^2$). Linear regression was performed with DBH ($mm$) as the independent variable to estimate the half-total leaf area for all trees on the site.

Further analyses of the LAI were made on a plot level. The resulting information
was summarized, filtering out dead trees to estimate the leaf area per plot ($m^2$). The
leaf area per plot ($m^2$) was then divided by the size in $m^2$ of the various plots to
obtain the LAI ($m^2m^{-2}$).
%math equation here
A regression model was created using stand density and basal area ($stemsha^{-1}$) as
the independent variables in equations 2 and 3. The variables are assumed to have a linear relationship with LAI:\\
%math equation here
Where:\\
$\beta_0$ = intercept\\
$\beta_1$ = slope\\
%math equation here
% Repeat former where statement here
The LAI ($m^2m^{-2}$) models created were compared to Goude et al., 2019 equations using MAE, MSE, RMSE, while AIC statistical metrics was used to test for the model having the best fit, between the two LAI models developed from this research. 

For biomass estimation, the dry weight of the sample needles and LAI needles were summed up to obtain the total sample branch needles weight ($g$) for each stratum (equation 4). The total sample branch needles weight (g) was then divided by the weight of the living branch for each stratum collected in the field to obtain the dry/fresh weight ratio (equation 5). The ratio was multiplied by the sum of the fresh weight ($g$) of the sample branch and the living branch of the stratum collected in the field to obtain the total weight of the needles for each stratum (equation 6). 

%equation here
% where statement here
%another equation here
% where statment

For the branches (living and dead), a similar procedure was followed. A dry and fresh weight ratio was calculated and multiplied by the sum of the samples and remaining branches. The result was summed up to obtain the total weight of the branches. The estimation of the stem weight followed almost the same procedure. The weight of the stem discs without bark was upscaled to the stem to get the bark weight. All the dry weights, stem without bark, bark, needles, branches without needles, stem discs, and cones were summed up to get the aboveground biomass of a tree. A linear regression equation was created using DBH ($mm$) as the independent variable and total weight ($kg$) as the dependent variable. The equation was applied to the site data, and the biomass of all the trees on the site was calculated. The result was summarized according to the plot to get the total weight ($kg$) in a plot and a regression equation using basal area, stand density ($stems ha^{-1}$) as the independent variables and the total weight per plot as the dependent variable.
\chapter{Results}
\section{Projected Area and Specific Leaf Area}
The highest projected area (PA) of LAI needles was found at the top of the crown, with mean values of 118 $cm^2$ for stratum three, 85 $cm^2$ for stratum two, and 67 $cm^2$ for stratum one needles (Table 1). The mean SLA per tree was 67.67 $(±1.902)$ $cm^2g^{-1}$ (mean + SE), with a significant difference between the strata ($p<0.0001$). Higher SLA was observed at the base of the crown and lower SLA at the top of the crown (Fig. 5). Projected area reduces from top of the crown to the bottom, while SLA decreases from the base to the top of the crown.
\section{Leaf Area}
There was a significant difference in LA between the individual trees for varying DBH ($p>0.001$) (Table 2). The individual tree regression was created using the DBH as the independent variable (Table 2). DBH and LA were positively correlated ($R2 = 0.883$), with large trees having a higher LA. The LA model in this study generally underestimated the LA with a standard error (SE) of $±2.48$, while Goude \textit{et al.} (2019)’s model overestimated LA, SE ±2.85 (Fig. 6). The RMSE, MSE and MAE were higher in Goude’s estimates than in this study's estimates (Fig. 7). The LA model developed in this study performs better than Goude \textit{et al.} (2019)’s model but tends to underestimate (Fig. 7).
\section{Leaf Area Index}
Directly measured LAI showed a positive correlation ($R2 = 0.55$) and a significant difference between varying stand densities ($p<0.001$). Directly measured LAI and basal area were also positively correlated ($R2 = 0.997$) (Table 3). Stand density and basal area were used to create a model function (Table 3). A simple multiple linear regression was used instead of a log-transformed linear regression model or a nonlinear regression model to fulfil the assumption of linearity between the variables. Supplementary Fig. 1-3 shows the goodness-of-fit for the multiple linear regression. This implies that the model assumptions and predictions aligns with the observed data.
\subsection{Model Comparison}
The quality of the models assessing how well they fit the data, was evaluated using the AIC. The $LAI_{model1}$, with stand density as its independent variable, has a poor fit to the data ($AICc = 1.57$), when compared with LAImodel2, which had a better fit ($AICc = -80.12$) (Table 4). The lower the AICc value of the models, the better the model quality. LAImodel2 boasts for more than 99\% the probability that it is the model with the best fit ($cum.wt = 1.00$, $AICcwt = 1$), Table 4. The better of the models developed is $LAI_{model2}$ with basal area as its explanatory variable.
\subsection{Study Model vs Goude et al., 2019 LAI Model}
The RMSE, MSE, and MAE were used to evaluate the accuracies and performance of the models developed in this study compared to Goude \textit{et al.} (2019)’s LAI model for \textit{P. sylvestris} (Fig. 8). The models from this study provided lesser error compared to Goude \textit{et al.} (2019)’s model, with a mean LAI of 2.92 ($±0.076$) $m^{2}m^{-2}$ (mean±SE), LAImodel1 had a mean LAI of 2.26 ($±0.056$) $m^2m^{-2}$, $LAI_{model2}$ had a mean of 2.26 ($±0.07$) $m^2m^{-2}$. All the LAI models developed in this study demonstrated improved performance and lesser errors than Goude \textit{et al.} (2019)’s model. There was no significant difference ($p > 0.05$) between directly measured LAI when compared with indirectly measured LAI estimated using Goude \textit{et al.} (2019)’s coefficients (Fig. 9). There was a weak positive correlation between directly measured LAI and indirectly measured LAI ($R2 = 0.1339$).
\section{Biomass Model}
The total aboveground biomass (kg) of the individual trees was calculated by summing up their components. The total aboveground biomass model created for \textit{P. sylvestris} was based on tree DBH ($mm$), Table 5. The independent and dependent variables were log-transformed. Total weight and DBH were positively correlated ($R2 = 0.86$) with a SE of $±11.17$.
% table5
At the stand level, the standing biomass of the site is 166 tonnes. Two biomass models were created using stand density ($Biomass_{model1}$) and basal area ($Biomass_{model2}$) as the independent variable. Stand density showed moderate positive correlation with the total aboveground stand biomass (R2 = 0.5), while basal area showed strong positive correlation ($R2 = 0.75$) (Table 6). There was a significant difference between the independent variables, stand density ($p< 0.05$) and basal area (p<0.0001) (Table 6). $Biomass_{model2}$ is having a higher coefficient of determination ($R2 = 0.75$). This shows that about 75\% of the variance of aboveground stand biomass is predictable when basal area is used as the independent variable in the model.
% table 6
Model performance was evaluated using RMSE and MAE for the biomass models. $Biomass_{model2}$ gave lesser error ($RMSE = 2143.3$, $MAE = 1726.1$) compared to Biomassmodel1 ($RMSE = 2356.2$, $MAE = 1668.5$) (Fig 10). The closer MAE and RMSE are to zero the more accurate the predictions to the real value.
% fig 10

\chapter{Discussion}
\section{LA and LAI models}
This research aimed to create site-specific LA and LAI models for P. sylvestris and then compare them with Goude \textit{et al.} (2019) model. LAI remains a vital metric providing stand productivity information (Lovynska \textit{et al.} 2018). The result from this study showed that LA could be estimated from DBH and LAI estimated from stand density or basal area.

Comparing the LA model from this research and Goude’s LA showed that this research model mostly underestimated LA, while Goude \textit{et al.} (2019)’s LA model generally overestimated (Fig. 6). LA model from this study estimates LA with minimal error compared to Goude \textit{et al.} (2019) LA, with a minimal error margin between both models (Fig. 7). The LA model of this study has lesser deviations from the actual values (Fig. 7) with a 9\% prediction accuracy compared to Goude \textit{et al.} (2019) function. The lesser deviation by the model of this research could indicate that Goude’s function was slightly improved. Goude \textit{et al.} (2019)’s generalized LA model gives reasonable estimates and can be applied when site-specific LA models are unavailable. This reasonable estimation by Goude’s LA model can be due to the study's large samples obtained across various locations in Sweden. The model from this study could be improved if height data of the site were available, then predictions are better than using only DBH. Generally, models where DBH is not the only predictor but are used with other tree variables, like crown depth, crown volume, and tree height, give better estimates than models with DBH alone (Xiao \textit{et al.} 2006). Given this, LA can be determined by multiple tree characteristics (Xiao \textit{et al.} 2006). However, other tree variables stated cannot be easily measured and getting such data at stand level can be expensive. Therefore, DBH, a very easy-to-measure tree variable, is a good predictor for LA models, given its significant relationship with LA (Table 2).

LAI in this study is high, 2.26 $m^{2}m^{-2}$, compared to Lovynska \textit{et al.} (2018) LAI with a value of 1.12 $m^{2}m^{-2}$ for a stand of 56 – 90 years, and that of Xiao \textit{et al.} (2006) with a value of 1.53 $m^2m^{-2}$ for a 73-year-old stand. Given that the stand is 24 years old, this could be a reason for its high LAI value. An LAI of 2.08 $m^{2}m^{-2}$ was estimated in a young \textit{P. sylvestris} stand of 17 years which agrees with findings from this study (Andrzej \& Kalucka 2008). According to Gholz \& Fisher (1982), LAI reached its maximum for Pinus contorta between 12 to 40 years, and they believe that the peak LAI could last about 30 years. However, this peak stage and how long it lasts depends on many factors like site nutrients, water availability, shade tolerance, and growth rate of the species (Vose \textit{et al.} 1994; White \textit{et al.} 2010).

When developing LAI functions, tree age is one of the parameters not usually considered. Only a few research studies have been carried out on the development of LAI and LA for \textit{P. sylvestris} over time (Vose \& Swank 1990; Andrzej \& Kalucka 2008). Parameters commonly used at the stand level for estimating LAI are basal area, stand density, and biomass (Vose \& Swank 1990; Vose \textit{et al.} 1994; Andrzej \& Kalucka 2008; Goude \textit{et al.} 2019). Site-specific LAI coefficients were developed for this research and compared with Goude \textit{et al.} (2019) LAI function. Indirectly measured LAI using Goude \textit{et al.} (2019) LICOR functions were also validated. $LAI_{model1}$ and LAImodel2 gave reasonable estimates with little error (Fig. 5). Previous studies have shown that LAI and stand density are related (Vose \& Swank 1990; Vose \textit{et al.} 1994; Andrzej \& Kalucka 2008), which also shows the influence of thinning on stand LAI (Guiterman \textit{et al.} 2012). Guiterman \textit{et al.} (2012) experiment to determine the influence of thinning on LAI showed that LAI is generally reduced and significantly different from an unthinned stand, but the thinning grade does not influence LAI. However, the case is different at the tree level, with trees in highly thinned stands expected to have increased LA and LAI (Guiterman \textit{et al.} 2012).
The error from $LAI_{model2}$ was minimal, having a near-perfect correlation ($R2 =
0.997$). This near-perfect correlation is because the basal area and LA functions are
measures dependent on DBH. This research was conducted at the site level, and
differences due to block effects were not explained. This relationship could not show the variation between the blocks, given that treatments (thinning grades) are not applied yet, thus the reason for a close-to-perfect relationship. 

When LAI was compared with indirectly measured LAI using \textit{Goude et al.} (2019)’s function, there was no relationship between the directly measured LAI and indirectly measured LAI (Fig. 6). The poorly explained relationship ($R2 = 0.1339$) between directly measured LAI (destructive sampling) and indirectly measured LAI indicates that Goude et al. (2019)'s function is probably not suitable to estimate LAI from LICOR-2200 for this experiment in Hallarp. Instead, it will be possible to use-site specific coefficients to improve Goude \textit{et al.} (2019) function and, thus, improve the relationship mentioned above.

\section{Biomass Model}
In this research, site-specific aboveground biomass models were also developed in addition to LA and LAI models. The most accurate method to determine trees' 30 biomass is through the destructive method, but this is time-consuming, and it is unwise to cut down all the trees in a stand to estimate their carbon content (Repola 2009; Durkaya \textit{et al.} 2016). Allometric relationships have been developed to reduce the forest's cost and level of destruction and get estimates of carbon stock in a stand (Repola 2009). The total dry weight ($kg$) of the tree was used as the response variable in contrast to the tree component-based allometric relationship for biomass that can be found in Repola (2009) and Repola and Ahnlund Ulvcrona (2014). Biomass estimation models developed for P. sylvestris in this study provides valuable insights into aboveground biomass estimation at both tree and stand levels. The individual tree biomass model based on DBH demonstrated a high positive correlation between the total dry weight and DBH, with an R2 value of $0.86$. This high correlation indicates that DBH is a reliable predictor for estimating the aboveground biomass of individual trees with a small margin of error (Mäkinen \& Isomäki 2004; West 2015; Durkaya \textit{et al.} 2016; Wegiel \& Polowy 2020). A drawback of this simplified allometric relationship involves its inability to account for the differences in the weight of tree components, like foliage and branches, whose amount and sizes are affected by intra-specific competition and silvicultural practices within a stand (Weiskittel \textit{et al.} 2011). 

At the stand level, models developed incorporated stand density and basal area as independent variables. Stand density and basal area are critical indicators of stand productivity and biomass accumulation (West 2015; Wegiel \& Polowy 2020). Another measure used as a reliable predictor of stand biomass is the dominant height or site index (Weiskittel \textit{et al.} 2011; West 2015). The result from this study confirmed the strong influence of both stand density and basal area on biomass estimation. The biomass model with basal area as its independent variable exhibited a highly significant influence on biomass estimation. This influence of the basal area on the total stand aboveground biomass is due to basal area giving an idea of the individual tree sizes and their growth rather than the number of trees in the forest. Stand density is an essential indicator of stand biomass at the initial stocking stage of any plantation, as the sizes of the seedlings do not matter. Stand density determines individual tree biomass and total biomass in a stand (Mäkinen \& Isomäki 2004). Highly stocked stands lead to increased stand biomass and reduced individual tree biomass due to competition for limited resources, while lowly stocked stands lead to reduced stand biomass and an increase in individual tree biomass (West 2014; Wegiel \& Polowy 2020).

It is important to acknowledge the limitation of this study, as the models developed in this research are specific to \textit{P. sylvestris} and limited to the study site. Other factors, such as site conditions, age, and tree characteristics, may influence biomass estimation and should be considered when applying the model to different stands or species

\chapter{Conclusion}
This research successfully developed site-specific LA, LAI, and aboveground biomass models for young \textit{P. sylvestris} stands. The LA and LAI models were compared with the existing generalized models of Goude \textit{et al.} (2019). The findings highlighted the importance of creating site-specific models, as they reduce prediction error. It also suggests the limitation of generalized models if their application is outside the range at which their sample data were collected. In this case, the age of the sample trees used could have led to a high margin of error, as age was not a parameter used in the generalized model.

Further research is recommended to refine and validate these models using larger sample sizes and incorporating stand conditions considering the influence of stand age, thinning and other environmental factors. The aboveground biomass model created for \textit{P. sylvestris} at both the tree and stand level offers a cost-effective and non-destructive alternative to estimate biomass, avoiding the need for destructive sampling, facilitating sustainable forest management, and contributing to carbon sequestration assessments while minimizing environmental impacts.

\end{document}